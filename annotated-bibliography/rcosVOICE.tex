\documentclass[12pt]{article}
\usepackage{graphicx} % Required for inserting images

\title{Eye tracking and its effectiveness in the assistance of nonverbal communication}
\author{Spencer Thys}
\date{October 2024}

\begin{document}
\maketitle

\section{Robust real time eye tracking for computer interface for
disabled people}
This paper demonstrates the mechanics of common eye tracking technology and shows an application it has for those with disabilities.
It explains the use of eye detection and its utilization of the reflectance power of the pupil in order to gain information from eye movement.
The method used shows the manipulation of images in order to highlight the eyes and isolate them from the rest of the human body.
The article also goes into the strengths of eye tracking, such as its non invasive nature
 
\section{Usability and Workload of Access Technology for People With Severe Motor Impairment: A Comparison of Brain-Computer Interfacing and Eye Tracking}
This paper shows that eye tracking can be used in order to select items on a screen using only the eyes.
The method used was to focus attention on what one wants to select and then close their eyes for 1.5 seconds.
The article highlights the application this has for those without motor function.
It is shown that there is inaccuracy with these selections, however, suggesting that there is lots of room for improvement such as faster and more accurate selection.
\section{Multimodal Emotion Recognition using EEG and Eye Tracking Data}
Description of paper goes here


\end{document}