\documentclass{article}

\usepackage[english]{babel}

\usepackage[letterpaper,top=2cm,bottom=2cm,left=3cm,right=3cm,marginparwidth=1.75cm]{geometry}

% Useful packages
\usepackage{xcolor}
\usepackage{amsmath}
\usepackage{graphicx}
\usepackage[colorlinks=true, allcolors=blue]{hyperref}

\title{Project VOICE: Findings as of 10/18}
\author{Spencer Thys}

\begin{document}
\maketitle
\begin{abstract}
\color{red}
Spencer Thys:
Eye-tracking devices
Real time eye-tracking on a computer interface
Looks like response times can be improved. Many experiments show that it takes a substantial amount of time to complete a task using eye tracking (at least a LOT longer than simply typing things into a search bar)
Emotions
Using pupil shape/dilation as an indicator of emotion is inconsistent but correlation is existent. This suggests improvements can be made (Perhaps with AI??)
Quadcopter control
Found an article that describes using EEG and eye tracking in conjunction in order to control a quadcopter… may reveal information about better controlling eye tracking in order to execute precise commands
Concepts:
To improve response time, I believe that the use of BCI may be very useful.
If we can create a way for one to send a consistent signal via BCI, we can pair that with eye tracking to quickly select items on an interface.
Sources:
https://docs.google.com/document/d/1IjoP99b9Tehzh5azujueZsX2O2syOGMM5J5x5kVFVOY/edit?tab=t.0 compiled sources… have not organized them into titles yet… sorry
Thoughts:
I think that AI has lots of potential not just in eye-tracking but in all nonverbal communication methods.
If AI is able to gather information on emotions and various brain signals, eye movements, etc… we may be able to turn smaller correlations into solid emotion detecting algorithms.
We may even be able to improve the accuracy of eye tracking using AI instead of spending more money on more precise tech.




\end{document}