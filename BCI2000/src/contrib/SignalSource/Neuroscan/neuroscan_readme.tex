\documentclass[letterpaper, oneside, 12pt]{article}
\usepackage{graphicx}
\usepackage{rotating}
\usepackage[american]{babel}

\selectlanguage{american}

\newcommand{\ie}{i.e.,}
\newcommand{\eg}{e.g.,}


% document
\begin{document}

%\include{title}
\title{Neuroscan Support in BCI2000}
\author{Gerwin Schalk\\ \small{\copyright{} 2004 Wadsworth Center, New
York State Department of Health}\\ \small{supported by work from
J\"{u}rgen Mellinger}\\ \small{Eberhard Karls University of T\"{u}
bingen, Germany}}
\maketitle

\tableofcontents

\newpage

%\begin{abstract}
%\end{abstract}

\section{Introduction}

\sloppypar \emph{NuAmps/SynAmps} are EEG recording systems from
Neuroscan, Inc., that are
widely used in clinical settings. The increasing use of BCI2000 in
those
environments created the need for Neuroscan support in BCI2000. This
document
describes this support that consists of two components: A BCI2000-
compatible
Source Module (\texttt{NeuroscanClient.exe}) and a command-line tool
(\texttt{neurogetparams}). These components are described below and
have been
tested with \emph{Neuroscan Acquire} version 4.3.1.

%\section{Background}
%As many other clinical systems, Neuroscan devices amplify and
digitize signals
%in a box outside of a computer, and then transmit these signals using
a
%proprietary and undocumented protocol to a computer.

\section{Neuroscan Source Module}

The BCI2000-compatible Source Module \texttt{NeuroscanClient.exe} can
be
used instead
of any other source module. In addition to standard parameters (\ie{}
\emph{SampleBlockSize}, \emph{SamplingRate}, \emph{SoftwareCh}), it
only
requests one Neuroscan-specific parameter (\emph{ServerAddress}) that
defines
the IP address (or host name) and the port number of the
\emph{Acquire} server.
An example for an appropriate definition is \texttt{localhost:3999}.
Before
starting BCI2000, you need to start \emph{Acquire}, click on the
\emph{S} symbol
in the top right corner (to enable the server) and start the server on
any port.
Please note that by default, \emph{Acquire} suggests port 4000, which
is a port
used by BCI2000. You might use port 3999 instead. Once the server is
running,
you can start BCI2000. It is imperative that certain parameters (\eg{}
the
number of channels) in BCI2000 match the settings in \emph{Acquire}.
You can
read these settings from \emph{Acquire} using the command line tool
described in
the following section. This command line tool can create a parameter
file
fragment that needs to be loaded on top of an existing parameter file
every time
a setting in \emph{Acquire} is changed.

\section{The \texttt{neurogetparams} Command Line Tool}

This command line tool reads system settings from the \emph{Neuroscan
Acquire}
server, displays them on a screen and creates a BCI2000 parameter file
fragment
if desired. It can be used as follows: \texttt{neurogetparams -address
localhost:3999 -paramfile test.prm}. (The \emph{Acquire} server has to
be
enabled before using this tool.) Once BCI2000 is configured correctly,
this
parameter file fragment needs to be loaded on top of the existing
configuration
to make sure that the settings match. You only need to repeat this
procedure if
you change settings in \emph{Acquire} (\eg{} such as the number of
channels or
the amplification).

\begin{verbatim}
BCI2000 Parameter Tool for Neuroscan Acquire V4.3
*******************************************************************
(C)2004 Gerwin Schalk and Juergen Mellinger
        Wadsworth Center, New York State Dept of Health, Albany, NY
        Eberhardt-Karls University of Tuebingen, Germany
*******************************************************************
Signal Channels: 32
Event Channels:  1
Block Size:      40
Sampling Rate:   1000
Bits/Sample:     16
Resolution:      0.168uV/LSB
Parameter file test.prm successfully written
\end{verbatim}


\section{Undocumented Communication}

\emph{Neuroscan Acquire} supports communication options that are not
documented
in the \texttt{NetAcquire.doc} document. BCI2000 needs to use two of
them, the
request for basic Neuroscan configuration information (which will
return
information such as the number of channels and the sampling rate) and
the
receipt of packets of data, to function properly. They are listed
below:\\

\begin{tabular}{|l|l|l|l|}
 \hline
 \textbf{Packet ID} & \textbf{Code} & \textbf{Request}   &
 \textbf{Body Size} \\
 \hline
 \hline
 "CTRL"   & 3 (Client Ctrl Code)    & 5 (Req Basic Info) & 0\\
 "DATA"   & 1 (Datatype Info)       & 3 (InfoType BasicInfo) & bodyLen
 \\
 "DATA"   & 2 (EEG Data)            & 1 (Raw 16-bit)     & bodyLen \\
 \hline
\end{tabular}

\subsection{Request Basic Information}

The client sends a request for basic information ("CTRL", Client Ctrl
Code, Req
Basic Info). The server responds by sending a data block ("DATA",
Datatype Info,
InfoType BasicInfo) that contains a data structure describing the
current
configuration of data acquisition in Acquire. This data structure is
as follows:

\begin{verbatim}
struct AcqBasicInfo
 {
 DWORD dwSize;      // Size of structure, used for version control
 int   nEegChan;    // Number of EEG channels
 int   nEvtChan;    // Number of event channels
 int   nBlockPnts;  // Samples in block
 int   nRate;       // Sampling rate (in Hz)
 int   nDataSize;   // 2 for "short", 4 for "int" type of data
 float fResolution; // Resolution for LSB
 };
\end{verbatim}

The first element, dwSize, can and should be cross checked with the
length of
the structure \texttt{AcqBasicInfo} and must match.

\subsection{Receive 16-bit Data Packet}

The client may send a \emph{Request to Start Sending Data} to the
Acquire server
(as described in the \texttt{NetAcquire.doc} document). This will
prompt the
server to start sending blocks of raw signal values. In order to
interpret these
signal values, the client must first have sent a \emph{Request for
Basic
Information}, as described above. The incoming data packets ("DATA",
EEG Data,
Raw 16-bit) can then be interpreted using this information and the
code below.

\begin{verbatim}
// make sure the body has the right length
assert((int)(pMsg->m_dwSize) ==
              (int)(m_BasicInfo.nEegChan+m_BasicInfo.nEvtChan)
              *m_BasicInfo.nBlockPnts*m_BasicInfo.nDataSize);
// process raw 16 bit data
for (int samp=0; samp<m_BasicInfo.nBlockPnts; samp++)
 {
 for (int ch=0; ch<m_BasicInfo.nEegChan+m_BasicInfo.nEvtChan; ch++)
  {
  short *sample=(short *)pMsg->m_pBody;
  printf("%d ",
  sample[samp*(m_BasicInfo.nEegChan+m_BasicInfo.nEvtChan)+ch]);
  }
 }
\end{verbatim}


%\begin{figure}[ht]
% \centerline{\scalebox{1}{\includegraphics{figure/environment.pdf}}}
% \caption{Graphical Environment.}
% \label{fig:environment}
%\end{figure}


\end{document}



